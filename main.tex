\documentclass[12pt]{article}
\usepackage{sbc-template}
\usepackage{graphicx,url}
\usepackage[brazil]{babel}   
\usepackage[utf8]{inputenc} 
\usepackage{listings}

\sloppy

\title{Formas de Obter o Número PI Computacionalmente}
\author{Matheus Mello de Lima}
\address{Professora: Cristiane Oliveira de Faria\\Cálculo IV - Ciência da Computação\\Universidade do Estado do Rio de Janeiro (UERJ)}
\date{June 2025}

\begin{document}

\maketitle

\begin{resumo} 
  Este trabalho tem como objetivo explorar diferentes formas de obter o número  \begin{math}\pi\end{math} (Pi) computacionalmente utilizando sete abordagens clássicas: série de Leibniz (Alternada), série de Nilakantha, fórmula de Machin (1706), série de Euler (Basel Problem), fórmula de Ramanujan, algoritmo de Chudnovsky, e a fórmula de Bailey–Borwein–Plouffe (BBP).
\end{resumo}

\section{Introdução Histórica}

Ao longo da história, diversos matemáticos contribuíram com séries e algoritmos que buscam uma aproximação precisa do número \begin{math}\pi\end{math} (Pi), além de precisos, esses métodos impulsionaram o desenvolvimento de conceitos matemáticos, servindo de base para avanços na análise numérica e na ciência da computação.

Este trabalho analisa computacionalmente sete métodos para o cálculo de \begin{math}\pi\end{math} (Pi), comparando precisão, convergência ao utilizar diferentes tipos de variáveis (\textit{float} e \textit{double}).

\subsection{Série de Leibniz}

Gottfried Wilhelm Leibniz (1646 -- 1716) foi um filósofo e matemático alemão responsável por conceber ideias de Cálculo Diferencial e Integral. Até hoje, a notação de Leibniz é a forma convencional de escrever trabalhos matemáticos de Cálculo.

Ele é considerado personalidade importante em diversas áreas de estudo com grandes contribuições à física e à tecnologia, e antecipou noções que surgiram muito mais tarde na teoria da probabilidade, biologia, medicina, geologia, psicologia, linguística e ciência da computação. A existência de pessoas com vasta experiência em diversas áreas, como Leibniz, se tornou cada vez mais raras na história da humanidade, principalmente após a Revolução Industrial (1760 –- 1840) e a disseminação da mão de obra especializada, por esta razão Leibniz é conhecido como o último gênio polímata.

Apesar da Série de Leibniz carregar seu nome, Leibniz não foi o primeiro a descobrir esta série de Pi. Ela foi descoberta pela primeira vez pelo matemático indiano Mādhava de Sangamagrāma (1340 –- 1425). Mādhava, fundador da Escola de Astronomia e Matemática de Kerala, foi um matemático e astrônomo indiano com contribuições pioneiras ao estudo de séries infinitas e responsável pela descoberta de séries para as funções trigonométricas de seno, cosseno, arco-tangente e métodos para calcular a circunferência de um círculo. Mādhava de Sangamagrāma leva esse nome por ser natural de Sangamagrama, Reino de Cochim (atual Irinjalakuda, Kerala, Índia).

Redescoberta em 1671 de forma independente por James Gregory e por Leibniz em 1673, a série de Leibniz, também conhecida por Série de Mādhava-Leibniz, é dada por:

\[\pi=4\sum\limits_{k=0}^{\infty}\frac{(-1)^{k}}{2k + 1}\]

\subsection{Série de Nilakantha}

Keļallur Nīlakaṇṭha Somayāji (1444 -– 1544) foi um matemático e astrônomo da Escola de Astronomia e Matemática de Kerala. Nīlakaṇṭha trouxe discussões sobre em séries infinitas de funções trigonométricas e problemas de álgebra e geometria esférica, contribuindo para trabalhos do matemático indiano Aryabhata do século V.

A série de Nīlakaṇṭha é definida por:

\[\pi=3+\sum\limits_{k=1}^{\infty}\frac{4}{(2k)(2k+1)(2k+2)}\]

\subsection{Fórmula de Machin}

\[\frac{\pi}{4}=4\tan^{-1}(\frac{1}{5})-\tan^{-1}(\frac{1}{239})\]

\subsection{Série de Euler}

Embora não forneça \begin{math}\pi\end{math} (Pi) diretamente, Euler mostrou que pode ser usada para obter aproximações de \begin{math}\pi\end{math} (Pi).


\[\sum\limits_{k=1}^{\infty}\frac{1}{k^{2}}=\frac{\pi^{2}}{6}\]


\subsection{Fórmula de Ramanujan}

\[\frac{1}{\pi}=\frac{2\sqrt{2}}{9801}\sum\limits_{k=0}^{\infty}\frac{(4k)!(1103+26390k)}{(k!)^{4}(396)^{4k}}\]

\subsection{Algoritmo de Chudnovsky}

\[\frac{1}{\pi}=12\sum\limits_{k=0}^{\infty}\frac{(-1)^{k}(6k)!(545140134k+13591409)}{(3k)!(k!)^{3}(640320)^{3k+3/2}}\]

\subsection{Fórmula de Bailey–Borwein–Plouffe}

\[\pi=\sum\limits_{k=0}^{\infty}\frac{1}{16^{k}}(\frac{4}{8k+1}-\frac{2}{8k+4}-\frac{1}{8k+5}-\frac{1}{8k+6})\]

\section{Análise}

\begin{lstlisting}[language=Python]
import numpy as np
    
def incmatrix(genl1,genl2):
    m = len(genl1)
    n = len(genl2)
    M = None #to become the incidence matrix
    VT = np.zeros((n*m,1), int)  #dummy variable
    
    #compute the bitwise xor matrix
    M1 = bitxormatrix(genl1)
    M2 = np.triu(bitxormatrix(genl2),1) 

    for i in range(m-1):
        for j in range(i+1, m):
            [r,c] = np.where(M2 == M1[i,j])
            for k in range(len(r)):
                VT[(i)*n + r[k]] = 1;
                VT[(i)*n + c[k]] = 1;
                VT[(j)*n + r[k]] = 1;
                VT[(j)*n + c[k]] = 1;
                
                if M is None:
                    M = np.copy(VT)
                else:
                    M = np.concatenate((M, VT), 1)
                
                VT = np.zeros((n*m,1), int)
    
    return M
\end{lstlisting}

\end{document}
